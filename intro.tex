documentclass[a4paper]{article}

\usepackage[english, ukrainian]{babel}
\usepackage[utf8x]{inputenc}
\usepackage{amsmath}
\usepackage{graphicx}
\usepackage[colorinlistoftodos]{todonotes}

\begin{document}
    \title{Застосування лазерноіндукованих структур для поверхнево підсиленої спектроскопії комбінаційного розсіяння світла}
    \maketitle
    \section{Втуп}
        TODO
    \section{Теоретичні відомості}
        \subsection{Комбінаційне розсіювання світла}
            \begin{enumerate}
                \item{Визначення}
                    Вивчення процесів взаємодії світла з речовиною є ефективним методом якісного і кількісного аналізу речовини. Існує два основних
                    типи зазначеної взаємодії: поглинання або розсіяння фотонів на молекулах речовини. Розсіяне світло може характреризуватися
                    виключно тією ж частотою, що і падаюче, а може містити хвилі на нових, характерних для цієї речовини частотах . У першому випадку
                    говорять, що має місце пружне або Релеєвське розсіяння світла. Другий випадок є прикладом непружного розсіяння і носить назву
                    комбінаційного розсіяння світла. 
                \item{Історія відкриття}
                    Явище комбінаційного розсіяння світла було відкрите Л.І. Мандельштамом і Г.С. Ландсбергом у 1928 році. Дослідники ставили перед
                    собою завдання зафіксувати тонку структуру в спектрі світла, що розсіюється твердим тілом. Згідно з їхніми міркуваннями, тонка
                    структура повинна виникати внаслідок модуляції розсіюваного світла пружніми тепловими хвилями, що присутні у твердому тілі. У
                    якості досліджуваного твердотільного середовища було обрано монокристал кварцу. Інтенсивність розсіяного молекулами кварцу світла
                    складає тисячні долі інтенсивності збуджуючого світла, тому на першому етапі досліджень було необхідно знайти кристал без домішок,
                    зафіксувати молекулярне розсіяння якого не складало б труднощів. 
                    
                    На момент початку досліджень у 1925 році сам факт молекулярного розсіяння в кристалах був мало дослідженим. Вченим необхідно
                    було встановити критерії, які дозволили б відрізняти розсіяння на молекулах від розсіяння на випадкових включеннях. Підбір
                    необхідних зразків та розробка методики реєстрації молекулярного розсіяння були закінчені у 1927 році, після чого дослідники
                    взялися за пошук спектральних змін у розсіяному кристалами кварцу світлі. Отримані результати проведених дослідів відрізнялися від
                    очікуваних. Виявилося, що по обидва боки від спектральної лінії збудження виникають симетричні лінії, зміщення яких відносно
                    центральної значно перевищує очікуваний частотний зсув. Чимало часу було витрачено на те, щоб впевнитися у достовірності отриманих
                    результатів. Після багатьох експериментів у вже не було сумнівів, що спостережувані лінії не є артифактними, а
                    пов'язані з невідомим раніше явищем.

                    В першій же ш публікації присвяченій отриманим результатам автори роблять правильне припущення, щодо природи нового явища. Вони
                    пишуть, що спосережуваний ефект може бути спричинений збудженням певних власних інфрачервоних частот кварцу за рахунок зменшення
                    енергії розсіюваного світла.
                \item{Застосування}
                \item{Існуючі проблеми}
            \end{enumerate}
        \subsection{Підсилення комбінаційного розсіяння світла}
            \begin{enumerate}
                \item{Існуючі методи}
                \item{Поверхневе підсилення}
            \end{enumerate}
        \subsection{Лазерноіндуковані поверхневі структури}
            \begin{enumerate}
                \item {Суть явища}
                \item {Теоретичні моделі}
                \item {Застосування}
            \end{enumerate}
    \section{Методика експерименту}
    \section{Результати}
    \section{Висновки}         
        
            
\end{document}
