\documentclass[a4paper]{article}

\usepackage[english, ukrainian]{babel}
\usepackage[utf8x]{inputenc}
\usepackage{amsmath}
\usepackage{graphicx}
\usepackage[colorinlistoftodos]{todonotes}

\begin{document}
    \title{Застосування лазерноіндукованих структур для поверхнево підсиленої спектроскопії комбінаційного розсіяння світла}
    \maketitle
    \section{Втуп}
        TODO
    \section{Теоретичні відомості}
        \subsection{Комбінаційне розсіювання світла}
            \begin{enumerate}
                \item{Визначення}
                    Вивчення процесів взаємодії світла з речовиною є ефективним методом якісного і кількісного аналізу речовини. Існує два основних
                    типи зазначеної взаємодії: поглинання або розсіяння фотонів на молекулах речовини. Розсіяне світло може характреризуватися
                    виключно тією ж частотою, що і падаюче, а може містити хвилі на нових, характерних для цієї речовини частотах . У першому випадку
                    говорять, що має місце пружне або Релеєвське розсіяння світла. Другий випадок є прикладом непружного розсіяння і носить назву
                    комбінаційного розсіяння світла. 
                \item{Історія відкриття}
                    Явище комбінаційного розсіяння світла було незалежно відкрито двома групами дослідників у 1928 році. Перша група працювала у
                    Москві під керівництвом \mbox{Л.І. Мандельштама} і \mbox{Г.С. Ландсберга}. Ці вчені ставили перед собою завдання зафіксувати тонку
                    структуру в спектрі світла, що розсіюється твердим тілом. Згідно з їхніми міркуваннями, тонка структура повинна виникати внаслідок
                    модуляції розсіюваного світла пружніми тепловими хвилями, що присутні у твердому тілі. У якості досліджуваного твердотільного
                    середовища було обрано монокристал кварцу. Інтенсивність розсіяного молекулами кварцу світла складає тисячні долі інтенсивності
                    збуджуючого світла, тому на першому етапі досліджень було необхідно знайти кристал без домішок, зафіксувати молекулярне розсіяння
                    якого було б значно простіше.
                    
                    На момент початку досліджень, у 1925 році, сам факт молекулярного розсіяння в кристалах був мало дослідженим. Вченим необхідно
                    було встановити критерії, які дозволили б відрізняти розсіяння на молекулах від розсіяння на випадкових включеннях. Підбір
                    необхідних зразків та розробка методики реєстрації молекулярного розсіяння були закінчені у 1927 році, після чого дослідники
                    взялися за пошук спектральних змін у розсіяному кристалами кварцу світлі. Отримані результати проведених дослідів відрізнялися від
                    очікуваних. Виявилося, що по обидва боки від спектральної лінії збудження виникають симетричні лінії, зміщення яких відносно
                    центральної значно перевищує очікуваний частотний зсув. Чимало часу було витрачено на те, щоб впевнитися у достовірності отриманих
                    результатів. Після багатьох експериментів у вчених вже не було сумнівів, що спостережувані лінії не є артифактними, а
                    пов'язані з невідомим раніше явищем.

                    В першій публікації, присвяченій отриманим результатам, автори роблять правильне припущення, щодо природи нового явища. \mbox{Г.С.
                    Ландсберг} та \mbox{ Л.І. Мандельштам } зазначають, що спосережуваний ефект може виникати внаслідок збудження певних власних
                    інфрачервоних частот кварцу за рахунок зменшення енергії розсіюваного світла. У цій же ж роботі вчені наводять правильний вираз
                    для кількісного обрахунку співвідношеня між інтенсивностями стоксової та антистоксової компонент комбінаційного розсіяння.

                    У той самий період часу Калькутті працювала група Ч.В. Рамана та К.С. Крішнана, яка займалася дослідженням розсіяння світла у
                    рідинах. Вченими були проведені досліди з великою кількістю різних речовин і в кожній з них фіксувалося слабке додаткове свічення.
                    Очищення розчинів не призвело до зменшення ефекту, тому було зроблено припущення про те, що спостережуваний ефект є проявом
                    досі невідомого явища. Раман вважав, що це явище може бути певним оптичним аналогом нещодавно відкритого ефекту Комптона і вирішив
                    продовжити дослідження цього явища.

                    Результати повторних експериментів переконали дослідників у тому, що спостережуваний ефект справді є аналогом комптонівського. У
                    своїй публікації під назвою "Оптичний аналог комптон-ефекту" автори зазначають, що попередні візуальні спостереження вказують на 
                    відсутність залежності положення ліній зміненої частоти від речовини, а така поведінка є характерною саме для ефекту Копмтона.
                    Незважаючи на хибні трактування, отримані результати були проявами того ж самого ефекту, який був зафіксований у Москві. 
                    
                    Перше спостереження ефекту комбінаційного розсіяння світла відбулося практично одночасно. Мандельштам і Ландсберг зафіксували
                    комбінаційне розсіяння на тиждень раніше від Рамана і Крішнана. Але завдяки впевненості у достовірності своєї гіпотези та
                    наявності великої кількості експериментальних даних індійські вчені публікують статтю з результами власних  досліджень на декілька
                    місяців раніше ніж Мандельштам та Ландсберг. У 1930 році Раман отримав Нобелівську премію з фізики за відкриття комбінаційного
                    розсіяння світла. Без уваги нобелівського комітету залишилася робота радянських вчених. 
                \item{Застосування}
                \item{Існуючі проблеми}
            \end{enumerate}
        \subsection{Підсилення комбінаційного розсіяння світла}
            \begin{enumerate}
                \item{Існуючі методи}
                \item{Поверхневе підсилення}
            \end{enumerate}
        \subsection{Лазерноіндуковані поверхневі структури}
            \begin{enumerate}
                \item {Суть явища}
                \item {Теоретичні моделі}
                \item {Застосування}
            \end{enumerate}
    \section{Методика експерименту}
    \section{Результати}
    \section{Висновки}         
        
            
\end{document}
